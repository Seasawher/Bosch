\documentclass[10pt]{jsarticle}

\input{preamble.tex}

\begin{document}

\title{Siegfried Bosch「Algebraic Geometry and Commutative Algebra」}
\author{\url{https://seasawher.github.io/kitamado/} \\ @seasawher}
\date{\today}
\maketitle


\bfsubsection{6.8 Cor.7}
\barquo{
Furthermore, for any $f \in A$ such that $D(f) \subset U$, the canonical map $A \to B$ induces an isomorphism $A_f \to B_f$ via localization and, hence , using 4.3/2, a commutative diagram
\[
\xymatrix{
M \ts_A B \ar[r] \ar[d] & \calf(U) \ar[d] \\
M \ts_A B_f \ar[r]^{\grs_f} & M \ts_A A_f
}
\]
where $\grs_f $ is an isomorphism.
}
\begin{proof}
  First, we show $A_f \to B_f$ is a isomorphism. The following is the proof.
  \begin{align*}
    A_f &= \calo_X(D(f)) \\
    &= \calo_X|_U(D(f)) \\
    &= \calo_U(D(f)) \\
    &= B_f
  \end{align*}
\textblue{The second part remains to be solved.}
\end{proof}


\bfsubsection{7.1 The projective n-space}
\barquo{
Also note that the ring of global sections of the structure sheaf on $X=\P^n_R$ is given by the intersection
\[
\calo_X(X) = \bigcap_{i=0}^n A_i = \bigcap_{i=0}^n R\left[ \f{t_0}{t_i}, \cdots , \f{t_n}{t_i} \right] = R.
\]
}
\begin{proof}
  Consider a more general case. Let $X = \bigcup_i X_i$ be a scheme got by glueing $X_i$. And $\calo_X(X_i) = A_i$ is contained by same ring $B$.
  Then, we get $\calo_X(X) = \bigcap_i A_i$. Why?

  Let $\calb$ be a open basis of $X$, $\calb = \setmid{U \subset X}{\exists i \;  U \opsub X_i}$. For any $U \in \calb$, restrinction maps $\bigcap_i \calo_X(X_i) \to \calo_X(U)$ induce a map $\bigcap_i \calo_X(X_i) \to \llim \calo_X (U)$ by universality of the limit.

  And we get the inverse map $ \llim \calo_X (U) \to \bigcap_i \calo_X(X_i) $ by gluing. Considering the definition of  limit, we can use gluing axiom of structure sheaf $\calo_X$.
\end{proof}





%\begin{thebibliography}{1}
%\bibitem{雪江N2} 雪江明彦『整数論2 代数的整数論の基礎』(日本評論社, 2013)
%\bibitem{Bosch} Siegfried Bosch『Algebraic Geometry and Commutative Algebra』(Springer, 2013)
  %\bibitem{松坂} 松坂和夫『集合・位相入門』(岩波書店, 1968)
  %\bibitem{齋藤} 齋藤正彦『線型代数学』(東京図書, 2014)
  %\bibitem{内田} 内田伏一『集合と位相』(裳華房, 1986)
  %\bibitem{松村} 松村英之『可換環論』(共立出版, 1980)
  %\bibitem{藤崎} 藤崎源二郎『体とガロア理論』(岩波書店, 1991)
  %\bibitem{雪江3} 雪江明彦『代数学3 代数学のひろがり』(日本評論社, 2011)
  %\bibitem{Harris} Joe Harris『Algebraic Geometry』(Springer, 1992)
  %\bibitem{GW} Ulrich G\"{o}rtz, Torsten Wedhorn『Algebraic Geometry I : Schemes  with Examples and Exercises』(Springer, 2010)
  %\bibitem{Liu} Qing Liu『Algebraic Geometry and Arithmetic Curves』(Oxford University Press, 2002)
  %\bibitem{Wedhorn} Torsten Wedhorn『Manifolds, Sheaves, and Cohomology』(Springer, 2016)
  %\bibitem{Rotman} Joseph J.Rotman『An Introduction to Homological Algebra』(Springer, 2009)
  %\bibitem{MacLane} Saunders Mac Lane『Categories for the Working Mathematician』(Springer, 1971)
  %\bibitem{Riehl} Emily Riehl『Category Theory in Context』(Dover, 2016)
%\end{thebibliography}

\end{document}
